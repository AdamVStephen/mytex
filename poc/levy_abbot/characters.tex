% Plain \TeX exploration of characters and how to get backslashes (and hence commands)

% Levy references p208 (index for \char) and p.36 which introduces char and page 181 for registers

The initial basic font is cmr10 and in this font, there are 256 characters.

We can show lowercase `a' via code 97 which can be written in four forms.

\item 1. $\backslash $char{97} gives $\longrightarrow$ \char97
\item 2. $\backslash $char{'141} gives $\longrightarrow$ \char'141 \ using octal ($64+32+1$).
\item 3. $\backslash $char{"61} gives $\longrightarrow$ \char"61 \ using hex ($96+1$).
\item 4. $\backslash $char{`$\backslash$a} gives $\longrightarrow$ \char`\a \ extracting the value for a with operator left-quote backslash.

Note that the construction `$\backslash$ gives the value of the next character which is
why the final version works.

To render as a number the extracted value that the operator gives needs a macro that
can take a value and then show it.  One indirect solution would be via a new variable.

\newcount\charval
We have created a $\backslash$newcount$\backslash$charval new counter and will
demonstrate that it works using the current year.
\charval=2023
The current year as held in $\backslash$charval is \the\charval.
\charval=`\a
The value for {\it a} extracted via the operator and stored in charval is \the\charval.

% TODO: Exercise to attempt
% Without using the intermediate register, we can attempt `\a to extract directly.

\medskip
\hrule
My First Loop Attempt
\hrule

In the next paragraph, I attempt to iterate over the alphabet using a loop construction.
Note that this is not covered in {\it Levy} but a sound example can be taken the paper on 
{\it Sorting within \TeX} by Kees van der Laan.

\newcount\i\i=97
\newcount\j\j=\i\advance\j26
\newcount\alphapos\alphapos=1
\newcount\a\a=1
\newcount\v\v=1
\newcount\veegap\veegap=52

{\loop\ifnum \i<\j%
\line{%
% Inject prefill spaces to the tune of the alphabetical position
\a=1{\loop\ifnum \a<\alphapos \ \advance\a1\repeat}
\the\i \ \char\i %
\v=1{\loop\ifnum \v<\veegap \ \advance\v1\repeat}%
\char\i \ \the\i \ \the\veegap \ \the\alphapos \ \hfill }%
\advance\i by 1\advance\alphapos1\advance\veegap by -2\repeat}

Section 12.5 in Levy is important in defining {\it active} characters.  These are of
catcode 13 (or $\backslash$\active) and can be defined to be control sequences or macros.


\bye
