% Plain TeX - demonstrating some of the techniques from Levy book.

\hsize=6.0in

A zeroth paragraph to essay: a simple math-mode backslash $ \backslash $, 
%a typewriter font backslash {\tt\char`\} and no more than that.
%a typewriter font backslash \textbackslash and no more than that.
and a suprisingly complicated requirement to really grok character codes
and fonts in an attempt to output some nicely formatted back-slashes.
See also {\tt characters.tex} for a more in-depth treatment.
\smallskip
This first paragraph of text is to be typeset with $hsize=6.0in$ and no other modifications.
It is written on
several
lines
and 
so
in the editor, the paragraph structure is not entirely clear.
The
use
of newline as a space means that this has no impact.
The expected standard indentation style is that the first sentence is indented by 
parindent of 20 points.
If the newline is suppressed by a comment marker%
we will check if a space was inserted between the words {\it marker} and {\it we}.

One blank line in the file should result in a second paragraph.
We now use input to input the lorem.tex file.
\input lorem.tex

% On escaping escapes : a technique to get familiar with
% Mnemonic : BackSlashMaths : BSM \bsm
\def\bsm{$\backlash $}
% Mnemonic : BackSlashTypewriter : BST \bst
\def\bst{{\tt\char`\\}}

One aside before proceeding. If we want to render the commands that generated this
file and not have them interpreted as a control sequence, we need to recall that the
backslack character ($\backslash $) has a category code of zero and is the escape 
character.  Two recommended ways to yield a backslash are to use mathmode and the
\$$\backslash$backslash\$ control sequence, or to use the typewriter font and
use $\{$\bst tt\bst char{`}\bst\bst$\}$ noting that to print braces requires
mathmode and escape. i.e. \$\bst$\{$\$.

\bst

Here is a left quote on it's own ` 

For our next trick, we demonstrate comprehension of the left and right margins
%which are controlled by \\leftskip and \\rightskip.

\bye
