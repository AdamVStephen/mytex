\TeX Tips and Tricks

{\bf Read the log file.}

For example, trying to create a macro to render a backslash without using math mode
can generate a double left quote, not because the command sequence is incorrect,
but because if typewriter fonts are not installed and configured, the sequence
is rendered in cmr10 in which the ASCII character code for the backslash in
typewriter font is a left quote.  {\bf \it Hmmmmph.}

{\bf TIP: create both .dvi and .ps and compare both if nonsensical results are obtained.}

Font errors may be permissions related.

%kpathsea: Running mktexpk --mfmode / --bdpi 600 --mag 1+0/600 --dpi 600 cmtt10
%mktexpk: Permission denied
%kpathsea: Appending font creation commands to missfont.log.
%page: Warning: font `cmtt10' at 600x600 not found, trying `cmr10' instead
%page: Warning: cmtt10: checksum mismatch (expected 3756670072, got 1274110073)

https://tex.stackexchange.com/questions/334440/tex-live-installation-and-a-problem-of-getting-evince-to-show-dvi-files

The problem is with evince and apparmor

Braces and Symbols and Input vs Font Enconding

https://stackoverflow.com/questions/48972814/code-in-latex-for-braces
https://tex.stackexchange.com/questions/44694/fontenc-vs-inputenc

Look at the techniques show in {\it A Gentle Introduction to \TeX}

\def\\{\char92{}}          %%%%% backslash %%%%% 
\def\lb{\char'173{}}       %%%%% left brace %%%%% 
\def\rb{\char'175{}}       %%%%% right brace %%%%% 
\def\sp{\char32{}}         %%%%% special space symbol %%%%% 
 
\def\beginliteral{ 
\vskip\baselineskip 
\begingroup 
\tt 
\obeylines 
%{\obeyspaces\global\let =\ } 
\catcode`\@=0 
\parskip=0pt\parindent=0pt 
\catcode`\$=12\catcode`\&=12\catcode`\^=12\catcode`\#=12 
\catcode`\_=12\catcode`\~=12 
\def\par{\leavevmode\endgraf} 
\catcode`\{=12\catcode`\}=12\catcode`\%=12\catcode`\\=12 
} 
 
\def\endliteral{\endgroup} 
%%%%%%%%%%%%%%%%%%%%%%%%%%%%%%%%%%%%%%%%%%%%%%%%%%%%%%%%%%%%%%%%%%%%% 
 
Borrowing from Michael Doob (Uni Manitoba) having defined some brace macros
and also a literal environment we should be good to leftbrace ({\tt \lb}) and ({\tt \rb})
as long as we use {\tt typewriter font} otherwise the characters yield (\lb) and (\rb).

Next we set out a literal stanza

\beginliteral
This is literal stuff {}
% $ ^ @ @ 
@endliteral

Exercise for the writer and reader is to explain why the literal environment 
has to be defined by the use of the @ character (hint catcode escape)
and then how to make a literal environment to support literal at.


% TODO: why did I WRONGLY get this impression ?
%
%{\bf Don't abuse the comment marker}
%Be wary of setting a variable and then applying the comment marker without a space.
%The impact is that the line continuation will be included in the definition of
%the variable and invariably (boom, boom) the value on the RHS will not be what you
%require.  It probably will also give bad effects.
%
%For example, try using leftskip to indent a paragraph.  Here are a working and failing
%version that differ only by one character.
%
%{\leftskip=1.5in% this will not work
%Foo is a paragraph with a {\bf lot} of indentation.
%\par}
%
%{\leftskip=1.5in % this will work
%Foo is a paragraph with a {\bf lot} of indentation.
%\par}

\bye

